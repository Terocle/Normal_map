\section{Анализ предметной области}
\subsection{Введение в предметную область}

На сегодняшний день визуальная составляющая цифровых продуктов приобретает всё большее значение. Повышение реалистичности изображений при сохранении приемлемой производительности становится одной из приоритетных задач разработчиков графических систем и движков \cite{gambetta2021}. Одним из эффективных решений, позволяющих достигать впечатляющих визуальных эффектов без чрезмерной нагрузки на оборудование, является применение карт нормалей (normal maps).

Карта нормалей -- это изображение, в котором каждый пиксель содержит информацию о направлении вектора нормали, то есть перпендикуляра к поверхности в данной точке \cite{lukacs2020}. Благодаря этому методу можно имитировать сложный микрорельеф -- такие детали, как трещины, вмятины, неровности и шероховатости -- без необходимости увеличивать количество геометрических примитивов в модели. По сути, подобная карта трансформирует плоское изображение в визуально объёмный объект, влияя на то, как свет отражается от поверхности, и тем самым создавая ощущение глубины.

Использование карт нормалей активно применяется в:
\begin{itemize}
	\item разработке компьютерных игр;
	\item 3D-визуализации;
	\item графических редакторах и движках;
	\item симуляторах освещения;
	\item виртуальной и дополненной реальности.
\end{itemize}

Что касается способов генерации таких карт, они варьируются от ручной проработки в 3D-графических редакторах до автоматического построения на основе двумерных изображений. Один из распространённых методов -- использование алгоритмов обработки градиентов, которые позволяют оценить рельеф изображения и построить нормали по интенсивности яркости. Также в последние годы появляются нейросетевые подходы, способные восстанавливать форму объекта по его изображению, генерируя реалистичные нормали на основе обучающих выборок.

Инструмент для генерации и интерактивной визуализации нормалей решает сразу несколько практических задач: повышает реалистичность текстур, сокращает время подготовки игровых и визуализационных контентов, а также даёт пользователю наглядный и доступный способ анализа поверхностей. Автоматизация этого процесса делает создание нормал-карт простым и эффективным.
\subsection{Методы представления информации в 2D и 3D}

Реалистичное отображение объектов и поверхностей в компьютерной графике требует передачи сведений о форме и структуре модели \cite{reed2020}. Эти данные могут храниться в виде полноценной полигональной сети (меша) или в виде различных текстур -- карт высот, нормалей, кривизны и глубины. Каждый из этих подходов имеет свои плюсы: текстуры экономят ресурсы, а меш позволяет проводить наиболее точные расчёты освещения и взаимодействия объектов.
\subsubsection{Геометрия в трёхмерной графике}

В 3D-графике основным источником геометрической информации является сетка (меш), представляющая собой набор вершин, рёбер и граней, формирующих объёмный объект. Такая модель полностью описывает форму объекта, позволяя точно рассчитывать его освещение, тени, отражения и взаимодействие с другими объектами. Однако, высокая геометрическая детализация требует значительных вычислительных ресурсов.

Для оптимизации визуализации часто применяются текстурные карты, содержащие в себе информацию о поверхностных характеристиках объекта, которые могут быть использованы при рендеринге без увеличения числа полигонов \cite{thalmann2021}.

\subsubsection{Карта высот (Height Map)}

Карта высот -- это одноканальное (grayscale) изображение, в котором яркость пикселя указывает на высоту точки относительно плоскости. Такие карты широко используют при:
\begin{itemize}
	\item генерации террейна (рельефов);
	\item параллаксных эффектов;
	\item предварительного моделирования.
\end{itemize}

Хотя карта высот хорошо передаёт общий рельеф, она не содержит информации о направлениях нормалей и требует дополнительных вычислений для освещения \cite{martins2024}.
\subsubsection{Карта нормалей (Normal Map)}

Карта нормалей служит для кодирования ориентации поверхности в каждом пикселе текстуры путём хранения в RGB‑каналах трёх компонент вектора нормали. В красном канале (R) обычно записывается смещение по оси X, в зелёном (G) -- по оси Y, а в синем (B) -- по оси Z. 

В игровых движках и современных шейдерах этот приём активно используют в шейдерах освещения, при динамическом изменении освещённости и в рендеринге по принципам PBR, где визуальное качество критично, а нагрузка на GPU должна оставаться низкой.

Существуют два основных типа нормалей:
\begin{enumerate}
	\item Tangent Space Normal Map -- нормали определяются относительно поверхности объекта (чаще всего применяются в игровых движках).
	\item Object Space Normal Map -- нормали представлены в глобальной системе координат (используются в более специфичных задачах, например, при baking-процессах).
\end{enumerate}
\subsubsection{Карта глубины (Depth Map)}

Карта глубины, или depth map, показывает расстояние от наблюдателя (камеры) до каждой точки сцены \cite{distante2020}. Используется, например, при:
\begin{itemize}
	\item рендеринге теней (shadow mapping);
	\item генерации эффектов глубины резкости;
	\item симуляции прозрачности и полупрозрачности.
\end{itemize}
\subsubsection{Карта кривизны (Curvature Map)}

Менее распространённый, но полезный тип карты. В отличие от нормалей, curvature map указывает, насколько поверхность в каждой точке вогнута или выпукла. Такие карты часто комбинируют с нормалями: они помогают автоматически добавлять износ, ржавчину и другие детали при процедурном затенении. Различные виды карт используются в зависимости от требований сцены, но именно normal map остаётся оптимальным компромиссом между реализмом и производительностью \cite{ansari2024}. 

\subsection{Применение карт нормалей в игровых движках}

Современные игровые движки представляют собой мощные платформы для создания визуально насыщенных и технически сложных трёхмерных миров. Одной из важнейших задач визуализации является имитация деталей поверхности объектов без значительного увеличения количества полигонов, то есть без дополнительной геометрической сложности. Для решения этой задачи широко применяются карты нормалей, которые моделируют особенности освещения поверхности на основе нормалей, рассчитанных на плоском или малополигональном меше.

Карты нормалей позволяют достичь высокого уровня реализма в отображении текстурированных объектов при сохранении высокой производительности, что делает их незаменимыми в игровой индустрии. 
\subsubsection{Принцип работы}

В классической 3D‑графике направление нормали к поверхности служит основой всех вычислений освещённости: алгоритмы Phong, Blinn‑Phong, модель Ламберта и PBR задействуют этот вектор, чтобы понять, как свет отражается от объекта. Однако если поверхность геометрически плоская, но фактически обладает микрорельефом -- бороздами, порами или другими деталями -- стандартные методы освещения не учитывают эти особенности \cite{bruno2024}. Использование normal map позволяет заменить «грубую» сеточную нормаль на точечный вектор из текстуры, и именно он участвует в расчёте освещения. В результате создаётся впечатление, что свет взаимодействует с настоящим объёмом, хотя геометрия остаётся простой.
\subsubsection{Игровые движки}
\paragraph{Unreal Engine}

Unreal Engine. В Unreal Engine работа с картами нормалей реализована на высоком уровне. В редакторе материалов (Material Editor) нормальная карта подключается к входу Normal. Особенности:
\begin{enumerate}
	\item Поддержка как tangent space, так и world space нормалей.
	\item Возможность смешивания нескольких нормалей через BlendAngleCorrectedNormals.
	\item Инструменты Baking'a из высокополигональных моделей.
	\item Совместимость с PBR-шейдингом.
	\item Поддержка DX12 и Vulkan расширяет визуальное качество за счёт сложных шейдеров с картами нормалей.
\end{enumerate}
\paragraph{Unity}

В Unity карта нормалей подключается через материал с поддержкой Standard Shader или URP/HDRP. Возможности:
\begin{enumerate}
	\item Поддержка Normal Map слота в шейдерах.
	\item Использование в постобработке (добавление контуров, эффектов ударов и др.).
	\item Возможность программного изменения нормалей через shader graph.
	\item В URP и HDRP используется более сложная система освещения, где нормали играют ключевую роль в эффектах SSAO, reflections, SSS.
\end{enumerate}

\paragraph{Godot Engine}

Godot поддерживает карты нормалей в своих материалах:
\begin{enumerate}
	\item Поддержка в SpatialMaterial и ShaderMaterial.
	\item Требуется импорт карты с типом «Normal Map».
	\item В шейдерах используются функции NORMALMAP и TANGENT.
\end{enumerate}
\subsubsection{Роль в оптимизации}

Применение карт нормалей позволяет снизить сложность моделей в LOD-системах: при удалении камеры вместо тяжёлых геометрий используются более простые меши с нормал-картами, что экономит видеопамять, повышает частоту кадров и уменьшает нагрузку на GPU без потери визуального качества \cite{umbaugh2022}.
\subsubsection{Карты нормалей и PBR}

В Physically Based Rendering нормали критически важны, так как определяют микрофасетную структуру поверхности, что влияет на:
\begin{itemize}
	\item рассеяние и отражение света;
	\item BRDF модели (например, Cook-Torrance);
	\item корректное отображение материалов (металл, кожа, ткань).
\end{itemize}

\subsection{Текущие инструменты и библиотеки для генерации нормалей}
\subsubsection{Обзор подходов}

Генерация карт нормалей из двухмерных изображений может выполняться с использованием как специализированных программных решений, так и библиотек общего назначения \cite{tyagi2021}. Основной задачей таких инструментов является анализ текстуры (чаще всего в формате RGB или Grayscale) и построение по ней псевдорельефа, из которого вычисляются локальные нормали. Эти нормали затем кодируются в нормал-карту, которая визуально представляется как RGB-изображение, где каждый канал соответствует компонентам нормали в 3D-пространстве.

Существующие решения можно условно разделить на две группы:
\begin{enumerate}
	\item Графические программы и утилиты с пользовательским интерфейсом (GUI) — Photoshop, xNormal, CrazyBump, Materialize и др.
	\item Библиотеки и инструменты программного уровня (API/CLI) — OpenCV, PIL, NumPy, Blender Python API и пр.
\end{enumerate}

В этом разделе подробно рассмотрим наиболее популярные инструменты обеих групп, их возможности, преимущества и ограничения.
\subsubsection{Специализированные графические инструменты}
\paragraph{Adobe Photoshop (плагин Normal Map Filter)}

Photoshop, как универсальный графический редактор, позволяет создавать карты нормалей с помощью встроенного фильтра «Generate Normal Map…» в модуле 3D или через сторонние расширения, например NVIDIA NormalMap Filter. Интерфейс интуитивен -- при помощи ползунков пользователь регулирует масштаб рельефа, уровень детализации и степень сглаживания. Можно накладывать несколько слоёв нормалей и редактировать их по отдельности, а при необходимости запускать пакетную обработку через Actions. При этом Photoshop является платным продуктом, и без создания или покупки лицензии доступ к фильтру ограничен. Кроме того, без дополнительного скриптинга в нём сложно автоматизировать массовую генерацию карт, и сам редактор не работает напрямую с глубинными картами или 3D‑моделями.
\paragraph{xNormal}

xNormal -- бесплатная утилита, изначально задуманная для работы с высокополигональными 3D‑моделями, однако она умеет преобразовывать и двухмерные height‑map изображения в карты нормалей. Эта программа известна своей точностью при запекании нормалей из сложных моделей и поддержкой большого числа форматов. Скорость генерации высокая, а в дополнение к normal maps можно получить ambient occlusion и displacement‑карты. Интерфейс xNormal выглядит устаревшим, и встроенные средства редактирования ограничены, кроме того, для работы с обычными цветными текстурами необходимо сначала преобразовать их в градации серого.
\paragraph{CrazyBump}

CrazyBump -- коммерческий инструмент, который анализирует обычные изображения и сразу выдаёт несколько типов карт: нормалей, смещения (displacement), отражения (specular) и ambient occlusion. При изменении настроек глубины или направления воображаемого источника света результаты обновляются мгновенно в окне предпросмотра. Это удобно для оперативной работы, однако для пакетной обработки большого количества файлов придётся искать обходные пути или скрипты. При генерации сложных текстур без ручной корректировки точность может страдать.
\paragraph{Materialize}

Materialize -- бесплатный и открытый проект, который на базе одного исходного изображения создаёт сразу несколько карт для PBR‑рабочего процесса: высотную карту, normal map, карту гладкости (gloss) и ambient occlusion. Встроенный просмотр материалов на сфере или плоскости даёт представление о реальном поведении текстуры. Программа не поддерживает автоматическую обработку множества файлов, и каждый материал требует ручной настройки параметров. Тем не менее, она остаётся удобным решением для небольших проектов и экспериментов с различными PBR‑эффектами.
\subsubsection{Открытые библиотеки и инструменты для программистов}
\paragraph{OpenCV (Python/C++)}

OpenCV (Open Source Computer Vision Library) -- мощная библиотека компьютерного зрения, которая активно используется для обработки изображений \cite{matveev2023}. Через операции свёртки (filter2D) можно реализовать операторы Sobel, Prewitt, Scharr, а затем по их результатам сформировать нормали.

Преимущества:
\begin{enumerate}
	\item Бесплатна и кроссплатформенна.
	\item Большой выбор фильтров и операторов.
	\item Возможность интеграции в собственное приложение.
	\item Поддержка NumPy и визуализация в Python.
\end{enumerate}

Ограничения:
\begin{enumerate}
	\item Требует знания Python/C++ и основ обработки изображений.
	\item Нет готовой функции «создать normal map» — нужно писать самостоятельно.
	\item Нет визуального интерфейса (только программный).
\end{enumerate}
\paragraph{PIL / Pillow (Python Imaging Library)}

PIL -- библиотека для базовой работы с изображениями в Python. Через Pillow можно реализовать фильтры и градиенты, а затем вручную построить нормали на основе светотеней или карт высот \cite{baskar2023}.

Преимущества:
\begin{enumerate}
	\item Простота.
	\item Идеально подходит для обработки и создания текстурных изображений.
	\item Лёгкость в использовании совместно с NumPy.
\end{enumerate}

Ограничения:
\begin{enumerate}
	\item Нет встроенных методов генерации нормалей.
	\item Ограниченные возможности свёрток.
	\item Не подходит для сложных фильтров или PBR-пайплайна.
\end{enumerate}
\paragraph{Blender API (Python)}

Blender предлагает Python API, который позволяет программно получать нормали из моделей, либо конвертировать height map в normal map с использованием модификаторов и узлов (nodes) в нодовой системе материалов.

Преимущества:
\begin{enumerate}
	\item Работа с 3D-моделями и изображениями одновременно.
	\item Поддержка визуализации результата.
	\item Возможность рендеринга с применённой картой нормалей.
\end{enumerate}

Ограничения:
\begin{enumerate}
	\item Сложность API.
	\item Требует изучения Blender и его архитектуры.
	\item Не подходит для лёгких решений или автоматических пайплайнов без интеграции.
\end{enumerate}

\subsection{Существующие подходы к генерации карт нормалей}
\subsubsection{Градиентные методы}

Градиентные методы строятся на идее получения нормали на основе локальных изменений яркости изображения. Предполагается, что яркость отражает высоту рельефа (height map), и, соответственно, градиенты яркости могут использоваться для вычисления компонент вектора нормали \cite{yane2021}. Наиболее часто используемые операторы:
\paragraph{Оператор Собеля}

Оператор Собеля применяет двумерные свёрточные ядра для определения горизонтальных и вертикальных градиентов. Это один из самых популярных операторов благодаря своей сбалансированной чувствительности к шуму и эффективности.

Формулы:
\[
G_x =
\begin{bmatrix}
	-1 & 0 & +1 \\
	-2 & 0 & +2 \\
	-1 & 0 & +1
\end{bmatrix}
\quad , \quad
G_y =
\begin{bmatrix}
	-1 & -2 & -1 \\
	0 & 0 & 0 \\
	+1 & +2 & +1
\end{bmatrix}
\]

Особенности:
\begin{enumerate}
	\item Хорошо сглаживает шум за счёт весов.
	\item Часто используется как дефолт в системах генерации нормалей.
\end{enumerate}
\paragraph{Оператор Превитта}

Метод Превитта очень похож на Собеля, но использует более простые ядра без усиленных центральных значений. Он более чувствителен к шуму, но быстрее и проще в реализации.

Формулы:
\[
G_x =
\begin{bmatrix}
	-1 & 0 & +1 \\
	-1 & 0 & +1 \\
	-1 & 0 & +1
\end{bmatrix}
\quad , \quad
G_y =
\begin{bmatrix}
	-1 & -1 & -1 \\
	0 & 0 & 0 \\
	+1 & +1 & +1
\end{bmatrix}
\]

Особенности:
\begin{enumerate}
	\item Вычисляется быстрее, чем Собель.
	\item Подходит для простых текстур без избыточной детализации.
\end{enumerate}
\paragraph{Оператор Шарра}

Оператор Шарра -- улучшенная версия Собеля, предложенная для более точного приближения производных, особенно при поворотах и изгибах. Он даёт лучшие результаты по сравнению с Собелем в высокочастотных деталях.

Формулы:
\[
G_x =
\begin{bmatrix}
	-3 & 0 & +3 \\
	-10 & 0 & +10 \\
	-3 & 0 & +3
\end{bmatrix}
\quad , \quad
G_y =
\begin{bmatrix}
	-3 & -10 & -3 \\
	0 & 0 & 0 \\
	+3 & +10 & +3
\end{bmatrix}
\]

Особенности:
\begin{enumerate}
	\item Высокая точность градиента.
	\item Используется в приложениях, требующих повышенной детализации.
\end{enumerate}
\subsubsection{Алгоритмы на базе искусственного интеллекта}

В последние годы для генерации нормалей всё чаще применяют глубокие сверточные нейронные сети (CNN), обученные на обширных датасетах. Вместо поиска простых градиентов они анализируют сложные текстурные паттерны и способны предсказывать «физически правдоподобные» нормали даже в тех областях, где классические методы бессильны \cite{shafik2020}.

Одной из популярных архитектур служит U‑Net: благодаря симметричной структуре «энкодер–декодер» и наличию skip-соединений она сохраняет мелкие детали, эффективно выполняя преобразование «grayscale в normal map». Похожим образом можно задействовать сети типа ResNet, VGG или кастомные автокодировщики, но все они требуют большого объёма примеров «height map в normal map» для качественного обучения.

ИИ-подходы дают высокий уровень реалистичности: модель «понимает» тени, варианты отражения и особенности материала, создавая нормали там, где нет явного перепада яркости \cite{jehangiri2024}. Однако за это приходится расплачиваться серьёзными вычислительными затратами, необходимостью подготовки и разметки большого датасета, а также «чёрным ящиком» -- внутренние веса сети сложно настраивать вручную.

В таблице \ref{compar:table} приведено сравнение двух методов.

\begin{xltabular}{\textwidth}{|X|X|X|}
	\caption{Сравнительный анализ подходов\label{compar:table}}	\\ \hline
	Критерий  & \centrow  Градиентные методы & \centrow Методы ИИ \\ \hline
	\endfirsthead
	\continuecaption{Продолжение таблицы \ref{compar:table}}
	Критерий & \centrow Градиентные методы & \centrow Методы ИИ \\ \hline 
	\finishhead
	\centrow 1 & \centrow 2 & \centrow 3  \\ \hline 
	Точность & Средняя & Высокая при хорошем обучении  \\ \hline 
	Скорость  & Очень высокая & Низкая (время на обучение) \\ \hline 
	Настраиваемость & Высокая (операторы, сила) & Низкая \\ \hline 
	Гибкость & Средняя & Высокая \\ \hline 
	Потребность в данных & Нет & Высокая (нужен датасет) \\ \hline 
	Простота реализации & Простая & Сложная \\ \hline 
\end{xltabular}

Классические градиентные методы остаются наиболее доступным и понятным способом генерации нормалей, особенно в интерактивных пользовательских системах. Тем не менее, методы на основе ИИ открывают новые перспективы в генерации сложных нормалей из произвольных изображений, особенно в условиях отсутствия карты высот \cite{farinella2020}.