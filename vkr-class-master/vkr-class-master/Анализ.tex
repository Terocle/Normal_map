\section{Анализ предметной области}
\subsection{Введение в предметную область}

В современной игровой индустрии и сфере компьютерной графики особое внимание уделяется визуальному качеству изображений и эффективности их обработки \cite{gambetta2021}. Одной из ключевых технологий, позволяющей достичь реалистичной визуализации объектов при относительно низкой вычислительной нагрузке, являются карты нормалей (англ. normal maps).

Карта нормалей представляет собой специальное изображение, в котором каждый пиксель кодирует нормаль — вектор, перпендикулярный поверхности в данной точке. Это позволяет передавать информацию о мелком рельефе объекта (царапины, бугры, пористость) без необходимости увеличивать количество геометрических полигонов. Иными словами, нормали придают 2D-изображению иллюзию трёхмерной структуры, влияя на то, как свет взаимодействует с поверхностью.

Использование карт нормалей активно применяется в:
\begin{itemize}
	\item разработке компьютерных игр;
	\item 3D-визуализации;
	\item графических редакторах и движках;
	\item симуляторах освещения;
	\item виртуальной и дополненной реальности.
\end{itemize}

В контексте игровой графики карты нормалей особенно важны для оптимизации производительности \cite{lukacs2020}. Вместо того, чтобы моделировать каждый мельчайший элемент (например, детали текстуры стены или одежды), разработчик может использовать карту нормалей, которая визуально создаёт эффект сложного рельефа при помощи освещения. Это даёт визуально богатую картинку без излишней нагрузки на графический процессор.

Создание карт нормалей может происходить различными способами:
\begin{itemize}
	\item вручную художником с помощью 3D-редакторов;
	\item с использованием 2D-текстур и алгоритмов анализа градиентов;
	\item с применением нейросетевых моделей, которые восстанавливают форму из изображения.
\end{itemize}

Процесс генерации нормалей из 2D-изображения становится особенно актуальным при работе с реалистичными текстурами (кирпич, камень, ткань, металл), полученными из фотографий или сгенерированными вручную. Это позволяет быстро и эффективно улучшать визуальное качество сцены без обращения к сложной 3D-модели.

Таким образом, предметная область разработки программного инструмента для генерации и визуализации карт нормалей напрямую связана с задачами:
\begin{itemize}
	\item повышения реалистичности отображаемых текстур;
	\item сокращения времени подготовки контента для игр и приложений;
	\item предоставления пользователю доступного и наглядного инструмента анализа поверхностей.
\end{itemize}

Приложение решает актуальную задачу — делает процесс генерации нормалей автоматическим и быстрым.
\subsection{Методы представления геометрической информации в 2D и 3D}

Одной из основных задач компьютерной графики является реалистичное отображение объектов и поверхностей. Для этого необходимо передавать геометрическую информацию о форме и структуре объектов \cite{reed2020}. В зависимости от задач, эта информация может быть представлена в различных формах: от полигональных моделей до специализированных текстур, таких как карты высот, карты нормалей, карты кривизны и карты глубины. Каждый из этих форматов имеет свои особенности, преимущества и области применения.
\subsubsection{Геометрия в трёхмерной графике}

В 3D-графике основным источником геометрической информации является сетка (меш), представляющая собой набор вершин, рёбер и граней, формирующих объёмный объект. Такая модель полностью описывает форму объекта, позволяя точно рассчитывать его освещение, тени, отражения и взаимодействие с другими объектами. Однако, высокая геометрическая детализация требует значительных вычислительных ресурсов.

Для оптимизации визуализации часто применяются текстурные карты, содержащие в себе информацию о поверхностных характеристиках объекта, которые могут быть использованы при рендеринге без увеличения числа полигонов \cite{thalmann2021}.

\subsubsection{Карта высот (Height Map)}

Карта высот — это изображение, где яркость каждого пикселя соответствует высоте точки относительно базовой плоскости. Обычно представляется в виде одноканального (grayscale) изображения. Такие карты часто применяются для:
\begin{itemize}
	\item генерации террейна (рельефов);
	\item параллаксных эффектов;
	\item предварительного моделирования.
\end{itemize}

Хотя карта высот хорошо передаёт общий рельеф, она не содержит информации о направлениях нормалей и требует дополнительных вычислений для освещения \cite{martins2024}.
\subsubsection{Карта нормалей (Normal Map)}

Карта нормалей кодирует направление нормали (перпендикуляра к поверхности) в каждом пикселе текстуры. Это направление обычно представлено как RGB-значение, где:
\begin{itemize}
	\item R (красный) — смещение по оси X;
	\item G (зелёный) — по оси Y;
	\item B (синий) — по оси Z.
\end{itemize}

Такие карты позволяют визуально имитировать сложные детали поверхности (царапины, выступы, неровности), даже если сама геометрия объекта остаётся плоской. Это значительно снижает нагрузку на систему и используется в реальном времени в:
\begin{itemize}
	\item шейдерах освещения;
	\item системах динамического освещения;
	\item рендеринге PBR (Physically Based Rendering).
\end{itemize}

Существуют два основных типа нормалей:
\begin{enumerate}
	\item Tangent Space Normal Map — нормали определяются относительно поверхности объекта (чаще всего применяются в игровых движках).
	\item Object Space Normal Map — нормали представлены в глобальной системе координат (используются в более специфичных задачах, например, при baking-процессах).
\end{enumerate}
\subsubsection{Карта глубины (Depth Map)}

Карта глубины, или depth map, показывает расстояние от наблюдателя (камеры) до каждой точки сцены \cite{distante2020}. Используется, например, при:
\begin{itemize}
	\item рендеринге теней (shadow mapping);
	\item генерации эффектов глубины резкости;
	\item симуляции прозрачности и полупрозрачности.
\end{itemize}

Карта глубины не несёт информации о направлениях поверхности, поэтому не применяется напрямую для освещения.
\subsubsection{Карта кривизны (Curvature Map)}

Менее распространённый, но полезный тип карты. Она показывает, насколько вогнута или выпукла поверхность в каждой точке. Часто используется в сочетании с картами нормалей в процедурах затенения или при автоматическом добавлении износа, ржавчины и т.д.

Разнообразие методов представления геометрии в цифровом изображении позволяет гибко адаптироваться под задачи конкретной сцены или проекта \cite{ansari2024}. Однако среди всех видов карт, именно карты нормалей обеспечивают наилучший баланс между визуальным качеством и вычислительной эффективностью. Это делает их незаменимыми в индустрии игр и 3D-визуализации.

Данная выпускная работа сосредоточена на автоматизированной генерации именно карт нормалей из двумерных изображений, а также их интерактивной визуализации, что позволяет оценить, насколько реалистично свет и тени взаимодействуют с имитируемой поверхностью. 
\subsection{Применение карт нормалей в игровых движках}

Современные игровые движки представляют собой мощные платформы для создания визуально насыщенных и технически сложных трёхмерных миров. Одной из важнейших задач визуализации является имитация деталей поверхности объектов без значительного увеличения количества полигонов, то есть без дополнительной геометрической сложности. Для решения этой задачи широко применяются карты нормалей, которые моделируют особенности освещения поверхности на основе нормалей, рассчитанных на плоском или малополигональном меше.

Карты нормалей позволяют достичь высокого уровня реализма в отображении текстурированных объектов при сохранении высокой производительности, что делает их незаменимыми в игровой индустрии. В данном разделе рассматриваются принципы работы с картами нормалей в популярных игровых движках, таких как Unity, Unreal Engine, Godot и других.
\subsubsection{Принцип работы}

В традиционной 3D-графике нормаль к поверхности используется для вычисления того, как на неё падает свет. Это основа для алгоритмов освещения, таких как:
\begin{itemize}
	\item Phong shading;
	\item Blinn-Phong lighting;
	\item Lambert shading;
	\item PBR (Physically Based Rendering).
\end{itemize}

Если поверхность содержит сложный рельеф (борозды, поры, микродетали), но геометрически она плоская, традиционное освещение не будет учитывать эти детали \cite{bruno2024}. Карта нормалей решает эту проблему, подменяя геометрические нормали пиксельными, извлечёнными из текстурного изображения. Эти нормали изменяют результат освещения, создавая иллюзию объёма и глубины.
\subsubsection{Использование в игровых движках}

Unreal Engine. В Unreal Engine работа с картами нормалей реализована на высоком уровне. В редакторе материалов (Material Editor) нормальная карта подключается к входу Normal. Особенности:
\begin{enumerate}
	\item Поддержка как tangent space, так и world space нормалей.
	\item Возможность смешивания нескольких нормалей через BlendAngleCorrectedNormals.
	\item Инструменты Baking'a из высокополигональных моделей.
	\item Совместимость с PBR-шейдингом.
	\item Поддержка DX12 и Vulkan расширяет визуальное качество за счёт сложных шейдеров с картами нормалей.
\end{enumerate}
\paragraph{Unity}

В Unity карта нормалей подключается через материал с поддержкой Standard Shader или URP/HDRP. Возможности:
\begin{enumerate}
	\item Поддержка Normal Map слота в шейдерах.
	\item Использование в постобработке (добавление контуров, эффектов ударов и др.).
	\item Возможность программного изменения нормалей через shader graph.
	\item В URP и HDRP используется более сложная система освещения, где нормали играют ключевую роль в эффектах SSAO, reflections, SSS.
\end{enumerate}

Пример использования в Shader Graph:
\begin{enumerate}
	\item Нода "Sample Texture 2D" для карты нормалей.
	\item Нода "Normal Unpack".
	\item Подключение к "Normal" входу Master Node.
\end{enumerate}
\paragraph{Godot Engine}

Godot поддерживает карты нормалей в своих материалах:
\begin{enumerate}
	\item Поддержка в SpatialMaterial и ShaderMaterial.
	\item Требуется импорт карты с типом "Normal Map".
	\item В шейдерах используются функции NORMALMAP и TANGENT.
\end{enumerate}
\subsubsection{Примеры применения}

Карты нормалей используются практически в каждом элементе трёхмерных игр:
\begin{enumerate}
	\item Стены, пол, текстуры зданий — реалистичный камень, кирпич, дерево.
	\item Существа и персонажи — морщины, мышцы, шрамы, чешуя.
	\item Интерфейсы и эффекты — динамические искажения (например, следы пуль, эффекты ударов).
	\item Постобработка — на основе нормалей можно создавать дополнительные визуальные эффекты, такие как затенение, контуры, интерактивные искажения.
\end{enumerate}
\subsubsection{Роль в оптимизации}

С помощью карт нормалей достигается LOD (Level of Detail) — при удалении камеры используются низкополигональные модели с картами нормалей вместо высокополигональных объектов \cite{umbaugh2022}. Это:
\begin{itemize}
	\item экономит память;
	\item повышает частоту кадров;
	\item снижает нагрузку на GPU при сохранении визуального качества.
\end{itemize}
\subsubsection{Карты нормалей и PBR}

В Physically Based Rendering нормали критически важны, так как определяют микрофасетную структуру поверхности, что влияет на:
\begin{itemize}
	\item рассеяние и отражение света;
	\item BRDF модели (например, Cook-Torrance);
	\item корректное отображение материалов (металл, кожа, ткань).
\end{itemize}

Карты нормалей являются ключевым инструментом в современном игровом производстве. Их использование позволяет добиться высокой степени реализма при минимальных затратах ресурсов. Поддержка на уровне движков, графических API и стандартов материалов делает нормали фундаментальным элементом визуализации, без которого невозможно представить себе ни одну современную 3D-игру.
\subsection{Текущие инструменты и библиотеки для генерации нормалей}
\subsubsection{Обзор подходов}

Генерация карт нормалей из двухмерных изображений может выполняться с использованием как специализированных программных решений, так и библиотек общего назначения \cite{tyagi2021}. Основной задачей таких инструментов является анализ текстуры (чаще всего в формате RGB или Grayscale) и построение по ней псевдорельефа, из которого вычисляются локальные нормали. Эти нормали затем кодируются в нормал-карту, которая визуально представляется как RGB-изображение, где каждый канал соответствует компонентам нормали в 3D-пространстве (обычно в tangent space).

Существующие решения можно условно разделить на две группы:
\begin{enumerate}
	\item Графические программы и утилиты с пользовательским интерфейсом (GUI) — Photoshop, xNormal, CrazyBump, Materialize и др.
	\item Библиотеки и инструменты программного уровня (API/CLI) — OpenCV, PIL, NumPy, Blender Python API и пр.
\end{enumerate}

В этом разделе подробно рассмотрим наиболее популярные инструменты обеих групп, их возможности, преимущества и ограничения.
\subsubsection{Специализированные графические инструменты}
\paragraph{Adobe Photoshop (плагин Normal Map Filter)}

Photoshop, будучи универсальным графическим редактором, предлагает возможность генерации нормалей через встроенный фильтр "Generate Normal Map..." (в разделе 3D), а также с помощью сторонних плагинов (например, NVIDIA NormalMap Filter — устарел, но ещё используется).

Преимущества:
\begin{enumerate}
	\item Интуитивный интерфейс.
	\item Гибкие параметры: масштаб, детализация, сглаживание.
	\item Возможность наложения и редактирования нормали слоями.
	\item Поддержка batch-обработки через actions.
\end{enumerate}

Ограничения:
\begin{enumerate}
	\item Платная программа, требующая лицензии.
	\item Мало гибкости в автоматизации (если не использовать скрипты).
	\item Нет поддержки глубинных карт или 3D-моделей.
\end{enumerate}
\paragraph{xNormal}

xNormal — бесплатный инструмент, изначально предназначенный для генерации карт нормалей из высокополигональных 3D-моделей. Также поддерживает генерацию нормалей из изображений (height map → normal map).

Преимущества:
\begin{enumerate}
	\item Высокая точность при генерации из моделей.
	\item Поддержка множества форматов.
	\item Быстрая генерация.
	\item Поддержка карт других типов: ambient occlusion, displacement и др.
\end{enumerate}

Ограничения:
\begin{enumerate}
	\item Интерфейс устаревший.
	\item Ограниченные функции редактирования.
	\item Не поддерживает сложную генерацию из обычных изображений (только grayscale → normal).
\end{enumerate}
\paragraph{CrazyBump}

CrazyBump — популярный коммерческий инструмент, который анализирует изображение и генерирует карты: normal, displacement, specular, occlusion.

Преимущества:
\begin{enumerate}
	\item Мгновенный предпросмотр результата.
	\item Простота в использовании.
	\item Поддержка псевдо-3D освещения для оценки результатов.
	\item Возможность тонкой настройки глубины, направленности света, масштабов.
\end{enumerate}

Ограничения:
\begin{enumerate}
	\item Программа закрытая и платная.
	\item Ограниченная автоматизация.
	\item Низкая точность при работе с сложными структурами без ручной настройки.
\end{enumerate}
\paragraph{Materialize}

Materialize — бесплатная программа с открытым исходным кодом для генерации PBR-текстурных карт из одного изображения.

Преимущества:
\begin{enumerate}
	\item Полностью бесплатна.
	\item Генерирует не только нормали, но и карты высот, gloss, ambient occlusion.
	\item Реалистичный предпросмотр материала на сфере или плоскости.
	\item Хорошая точность генерации из grayscale-изображений.
\end{enumerate}

Ограничения:
\begin{enumerate}
	\item Нет поддержки автоматизации.
	\item Не подходит для потоковой обработки большого числа файлов.
	\item Требует ручной настройки каждого материала.
\end{enumerate}
\subsubsection{Открытые библиотеки и инструменты для программистов}
\paragraph{OpenCV (Python/C++)}

OpenCV (Open Source Computer Vision Library) — мощная библиотека компьютерного зрения, которая активно используется для обработки изображений \cite{matveev2023}. Через операции свёртки (filter2D) можно реализовать операторы Sobel, Prewitt, Scharr, а затем по их результатам сформировать нормали.

Преимущества:
\begin{enumerate}
	\item Бесплатна и кроссплатформенна.
	\item Большой выбор фильтров и операторов.
	\item Возможность интеграции в собственное приложение.
	\item Поддержка NumPy и визуализация в Python.
\end{enumerate}

Ограничения:
\begin{enumerate}
	\item Требует знания Python/C++ и основ обработки изображений.
	\item Нет готовой функции «создать normal map» — нужно писать самостоятельно.
	\item Нет визуального интерфейса (только программный).
\end{enumerate}
\paragraph{PIL / Pillow (Python Imaging Library)}

PIL — библиотека для базовой работы с изображениями в Python. Через Pillow можно реализовать фильтры и градиенты, а затем вручную построить нормали на основе светотеней или карт высот.

Преимущества:
\begin{enumerate}
	\item Простота.
	\item Идеально подходит для обработки и создания текстурных изображений.
	\item Лёгкость в использовании совместно с NumPy.
\end{enumerate}

Ограничения:
\begin{enumerate}
	\item Нет встроенных методов генерации нормалей.
	\item Ограниченные возможности свёрток.
	\item Не подходит для сложных фильтров или PBR-пайплайна.
\end{enumerate}
\paragraph{Blender API (Python)}

Blender предлагает Python API, который позволяет программно получать нормали из моделей, либо конвертировать height map → normal map с использованием модификаторов и узлов (nodes) в нодовой системе материалов.

Преимущества:
\begin{enumerate}
	\item Работа с 3D-моделями и изображениями одновременно.
	\item Поддержка визуализации результата.
	\item Возможность рендеринга с применённой картой нормалей.
\end{enumerate}

Ограничения:
\begin{enumerate}
	\item Сложность API.
	\item Требует изучения Blender и его архитектуры.
	\item Не подходит для лёгких решений или автоматических пайплайнов без интеграции.
\end{enumerate}

Существующие инструменты для генерации карт нормалей охватывают широкий спектр задач: от простого преобразования 2D-изображений до генерации нормалей с высокополигональных моделей \cite{baskar2023}.
\subsection{Существующие подходы к генерации карт нормалей}

Генерация карт нормалей из двумерных изображений — одна из ключевых задач в обработке изображений для компьютерной графики и геймдева. Существует два основных класса подходов:
Ограничения:
\begin{enumerate}
	\item Градиентные методы, основанные на классической обработке изображений.
	\item Методы на базе искусственного интеллекта (ИИ), использующие глубокие нейронные сети.
\end{enumerate}

Каждый из подходов имеет свои особенности, сильные и слабые стороны, а также различия по точности, скорости вычислений и качеству получаемых нормалей.
\subsubsection{Градиентные методы}

Градиентные методы строятся на идее получения нормали на основе локальных изменений яркости изображения. Предполагается, что яркость отражает высоту рельефа (height map), и, соответственно, градиенты яркости могут использоваться для вычисления компонент вектора нормали \cite{yane2021}. Наиболее часто используемые операторы:
\paragraph{Оператор Собеля}

Оператор Собеля применяет двумерные свёрточные ядра для определения горизонтальных и вертикальных градиентов. Это один из самых популярных операторов благодаря своей сбалансированной чувствительности к шуму и эффективности.

Формулы:
\[
G_x =
\begin{bmatrix}
	-1 & 0 & +1 \\
	-2 & 0 & +2 \\
	-1 & 0 & +1
\end{bmatrix}
\quad , \quad
G_y =
\begin{bmatrix}
	-1 & -2 & -1 \\
	0 & 0 & 0 \\
	+1 & +2 & +1
\end{bmatrix}
\]

Особенности:
\begin{enumerate}
	\item Хорошо сглаживает шум за счёт весов.
	\item Часто используется как дефолт в системах генерации нормалей.
\end{enumerate}
\paragraph{Оператор Превитта}

Метод Превитта очень похож на Собеля, но использует более простые ядра без усиленных центральных значений. Он более чувствителен к шуму, но быстрее и проще в реализации.

Формулы:
\[
G_x =
\begin{bmatrix}
	-1 & 0 & +1 \\
	-1 & 0 & +1 \\
	-1 & 0 & +1
\end{bmatrix}
\quad , \quad
G_y =
\begin{bmatrix}
	-1 & -1 & -1 \\
	0 & 0 & 0 \\
	+1 & +1 & +1
\end{bmatrix}
\]

Особенности:
\begin{enumerate}
	\item Вычисляется быстрее, чем Собель.
	\item Подходит для простых текстур без избыточной детализации.
\end{enumerate}
\paragraph{Оператор Шарра}

Оператор Шарра — улучшенная версия Собеля, предложенная для более точного приближения производных, особенно при поворотах и изгибах. Он даёт лучшие результаты по сравнению с Собелем в высокочастотных деталях.

Формулы:
\[
G_x =
\begin{bmatrix}
	-3 & 0 & +3 \\
	-10 & 0 & +10 \\
	-3 & 0 & +3
\end{bmatrix}
\quad , \quad
G_y =
\begin{bmatrix}
	-3 & -10 & -3 \\
	0 & 0 & 0 \\
	+3 & +10 & +3
\end{bmatrix}
\]

Особенности:
\begin{enumerate}
	\item Высокая точность градиента.
	\item Используется в приложениях, требующих повышенной детализации.
\end{enumerate}
\paragraph{Параметры глубины и силы нормалей}

Во всех градиентных методах можно управлять дополнительными параметрами:
\begin{enumerate}
	\item Сила нормали (Normal strength) — коэффициент, усиливающий значения градиента. При высокой силе — нормали становятся более контрастными, а эффект освещения более выраженным. При низкой — нормаль сглажена, и поверхность выглядит плоско.
	\item Глубина (Depth scale) — влияет на предполагаемую разницу высот между пикселями. Чем больше глубина, тем "рельефнее" будет восприниматься поверхность.
\end{enumerate}

Эти параметры критически важны для тонкой настройки визуального результата и дают пользователю возможность адаптировать карту нормалей под конкретные нужды — от гладких металлов до шероховатых поверхностей.

\subsubsection{Алгоритмы на базе искусственного интеллекта}

Современные методы генерации нормалей используют глубокие сверточные нейронные сети (CNN), обученные на больших датасетах. Они не зависят от простых градиентов, а анализируют сложные текстурные паттерны для предсказания более «физически достоверных» нормалей \cite{shafik2020}.

Архитектура U-Net изначально была разработана для сегментации изображений, но прекрасно адаптируется для задач перевода одного вида изображения в другой (image-to-image translation), например, grayscale → normal map.

Особенности:
\begin{enumerate}
	\item Использует симметричную архитектуру энкодер-декодер.
	\item Сохраняет детали за счёт skip-коннектов.
	\item Позволяет обучать модели, которые предсказывают более реалистичную геометрию.
\end{enumerate}

CNN общего назначения могут использоваться и другие архитектуры, такие как ResNet, VGG или кастомные autoencoder-сети. Они требуют большого количества данных для обучения (например, пар height map — normal map).

Преимущества ИИ-подходов:
\begin{enumerate}
	\item Высокая реалистичность результатов.
	\item Возможность «понимания» структуры изображения — тени, материалы, освещение.
	\item Генерация нормалей даже там, где нет явных градиентов.
\end{enumerate}

Недостатки:
\begin{enumerate}
	\item Требуется обучение на большом датасете.
	\item Высокая потребность в вычислительных ресурсах.
	\item Чёрный ящик — невозможно детально контролировать параметры.
\end{enumerate}

В таблице \ref{compar:table} приведено сравнение двух методов.

\begin{xltabular}{\textwidth}{|c|X|X|}
	\caption{Сравнительный анализ подходов\label{compar:table}}	\\ \hline
	Критерий  & \centrow  Градиентные методы & \centrow Методы ИИ \\ \hline
	\endfirsthead
	\continuecaption{Продолжение таблицы \ref{compar:table}}
	Критерий & \centrow Градиентные методы & \centrow Методы ИИ \\ \hline 
	\finishhead
	Точность & Средняя & Высокая при хорошем обучении  \\ \hline 
	Скорость  & Очень высокая & Низкая (время на обучение) \\ \hline 
	Настраиваемость & Высокая (операторы, сила) & Низкая \\ \hline 
	Гибкость & Средняя & Высокая \\ \hline 
	Потребность в данных & Нет & Высокая (нужен датасет) \\ \hline 
	Простота реализации & Простая & Сложная \\ \hline 
\end{xltabular}

Классические градиентные методы остаются наиболее доступным и понятным способом генерации нормалей, особенно в интерактивных пользовательских системах. Возможность выбирать оператор, настраивать силу и глубину нормалей делает такие подходы гибкими и пригодными для различных сценариев визуализации.

Тем не менее, методы на основе ИИ открывают новые перспективы в генерации сложных нормалей из произвольных изображений, особенно в условиях отсутствия карты высот. Их интеграция требует дополнительной подготовки, но может значительно расширить функциональность таких систем, как описываемое в данной выпускной работе приложение.
\subsection{Применение карт нормалей в игровой индустрии}

Карта нормалей (normal map) является неотъемлемым элементом современного визуального контента в видеоиграх. Она позволяет симулировать мелкие детали поверхности (царапины, поры, выпуклости, швы и т. д.) без увеличения количества полигонов, что значительно снижает нагрузку на графический процессор (GPU) и повышает реализм сцены \cite{jehangiri2024}.
\subsubsection{Оптимизация графики}

Одной из ключевых задач в разработке игр является оптимизация производительности, особенно для игр, работающих в реальном времени. Создание сложных геометрических поверхностей (например, барельефов или ткани) с помощью реальных 3D-полигонов приводит к сильной нагрузке на GPU.

Карта нормалей позволяет: 
\begin{itemize}
	\item заменить высокополигональную геометрию на плоскую модель с детализированной визуальной текстурой;
	\item снизить время отрисовки объектов (draw call);
	\item сохранить визуальную достоверность сцены.
\end{itemize}

Таким образом, с помощью нормалей можно отображать, например, кирпичную кладку, металлическую ржавчину, кожу или древесную текстуру, используя всего один плоский полигон и набор текстур.
\subsubsection{Свет и тени}

Карта нормалей напрямую влияет на взаимодействие материала с освещением. Вместо использования геометрических нормалей, GPU использует нормали из текстуры для расчета направления отражения света и генерации теней \cite{farinella2020}.

Результаты:
\begin{enumerate}
	\item Более реалистичное затенение объектов.
	\item Правильное преломление света от текстурных деталей.
	\item Эффекты бликов и рассеивания.
\end{enumerate}

Особенно важны карты нормалей при использовании динамического освещения, когда положение источника света может меняться во времени (например, фонарик в руках персонажа или проходящий мимо огонь).
\subsubsection{Технологии и движки}

Практически все современные игровые движки поддерживают работу с картами нормалей "из коробки".

Примеры:
\begin{enumerate}
	\item Unreal Engine: использует карты нормалей в материалах с поддержкой PBR (Physically-Based Rendering). Поддерживаются как Tangent-space, так и World-space нормали.
	\item Unity: аналогично, поддерживает текстуры нормалей в шейдерах Standard Shader. Также доступен пост-обработочный рендеринг с использованием нормальных буферов.
	\item Godot Engine, CryEngine, Source 2, Frostbite и другие — все они используют карты нормалей для повышения фотореализма без потери производительности.
\end{enumerate}
\subsubsection{Использование в PBR-материалах}

В рамках подхода PBR (физически корректного рендеринга), карта нормалей — один из ключевых слоёв вместе с:
\begin{itemize}
	\item альбедо (цветовая текстура);
	\item Roughness/Glossiness (шероховатость);
	\item Metallic (металличность);
	\item AO (ambient occlusion).
\end{itemize}

Normal map обеспечивает информацию о микрогеометрии и является обязательной для имитации реалистичного взаимодействия материала со светом. Без неё поверхность выглядит "мыльной" и плоской.
\subsubsection{Особенности в жанрах и платформах}

В зависимости от жанра игры и платформы карта нормалей может использоваться по-разному:
\begin{enumerate}
	\item Шутеры от первого лица (FPS): важны мелкие детали, особенно при близком расстоянии к объектам (оружие, стены, текстура кожи).
	\item RPG и стратегии: используют нормали на одежде, зданиях, ландшафте.
	\item Мобильные игры: карты нормалей используются реже, но адаптируются с меньшим разрешением для сохранения производительности.
	\item VR и AR: карты нормалей важны как средство сохранения качества при ограничениях на количество полигонов.
\end{enumerate}
\subsubsection{Примеры применения}

Технология широко распространена и применяется в различных играх:
\begin{enumerate}
	\item The Last of Us Part II (Naughty Dog): сложная система нормалей используется на коже, волосах, одежде и окружении. Детали читаются даже при близких ракурсах.
	\item DOOM Eternal (id Software): агрессивно применяются нормали для повышения резкости и глубины даже на высокоскоростных аренах.
	\item Alien: Isolation (Creative Assembly): нормали играют ключевую роль в реалистичном освещении корабля и создании атмосферы.
\end{enumerate}
