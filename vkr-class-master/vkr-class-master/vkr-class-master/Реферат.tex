\abstract{РЕФЕРАТ}

Объем работы равен \formbytotal{lastpage}{страниц}{е}{ам}{ам}. Работа содержит \formbytotal{figurecnt}{иллюстраци}{ю}{и}{й}, \formbytotal{tablecnt}{таблиц}{у}{ы}{}, \arabic{bibcount} библиографических источников и \formbytotal{числоПлакатов}{лист}{}{а}{ов} графического материала. Количество приложений – 2. Графический материал представлен в приложении А. Фрагменты исходного кода представлены в приложении Б.

Перечень ключевых слов: коммерческий сайт, Система, CMS, Битрикс, Joomla, аддитивные технологии, 3D-принтеры, услуги, сервисы, информатизация, автоматизация, информационные технологии, веб-форма,  Apache, классы, база данных, средства защиты информации, подсистема, компонент, модуль, сущность, информационный блок, метод, контент-редактор, администратор, пользователь, web-сайт.

Объектом разработки является web-сайт компании,  занимающейся производством 3D-принтеров, выпуском оборудования для создания порошков, разработкой программного обеспечения и организацией центров аддитивного производства.

Целью выпускной квалификационной работы является привлечение клиентов, увеличение заказов, информирование о продукции и услугах путем создания сайта компании.

В процессе создания сайта были выделены основные сущности путем создания информационных блоков, использованы классы и методы модулей, обеспечивающие работу с сущностями предметной области, а также корректную работу web-сайта, разработаны разделы, содержащие информацию о компании, ее деятельности, производимой продукции и услугах, разработан сервис по заказу 3D-деталей.

При разработке сайта использовалась система управления контентом "<1С-Битрикс: Управление сайтом">.

Разработанный сайт был успешно внедрен в компании.

\selectlanguage{english}
\abstract{ABSTRACT}
  
The volume of work is \formbytotal{lastpage}{page}{}{s}{s}. The work contains \formbytotal{figurecnt}{illustration}{}{s}{s}, \formbytotal{tablecnt}{table}{}{s}{s}, \arabic{bibcount} bibliographic sources and \formbytotal{числоПлакатов}{sheet}{}{s}{s} of graphic material. The number of applications is 2. The graphic material is presented in annex A. The layout of the site, including the connection of components, is presented in annex B.

List of keywords: commercial website, System, CMS, Bitrix, Joomla, additive technologies, 3D printers, services, services, informatization, automation, information technology, web form, Apache, classes, database, component, module, entity, information block, method, content editor, administrator, user, web site.

The object of the research is the analysis of information technologies for the development of a production company's website.

The object of the development is the website of a company engaged in the production of 3D printers, the production of equipment for the creation of powders, software development and the organization of additive manufacturing centers.

The purpose of the final qualifying work is to attract customers, increase orders, inform about products and services by creating a company website.

In the process of creating the site, the main entities were identified by creating information blocks, classes and methods of modules were used to ensure work with the entities of the subject area, as well as the correct operation of the website, sections containing information about the company, its activities, products and services were developed, a service for ordering 3D parts was developed.

When developing the site, the content management system <<1C – Bitrix: Site Management>> was used.

The developed website was successfully implemented in the company.
\selectlanguage{russian}
