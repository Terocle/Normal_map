\section*{ВВЕДЕНИЕ}
\addcontentsline{toc}{section}{ВВЕДЕНИЕ}

Цифровая обработка изображений -- это активно развивающееся направление, находящее применение в таких областях, как компьютерная графика, медицина, робототехника, системы машинного зрения и игровой индустрии. Одной из важных задач в данной сфере является построение карт нормалей -- изображений, содержащих информацию о направлениях поверхностей объектов. Такие карты позволяют придавать двумерным изображениям видимость трёхмерной структуры за счёт симуляции освещения.

Применение карт нормалей особенно актуально в разработке компьютерных игр и визуальных приложений, где важно создавать реалистичный светотеневой эффект при минимальных затратах вычислительных ресурсов. Они также используются в редакторах текстур, 3D-моделировании и интерфейсах визуального анализа изображений.

Настоящая работа посвящена разработке программной системы, позволяющей пользователю загружать изображения, генерировать на их основе карты нормалей с помощью алгоритмов градиентного анализа и визуализировать результат с возможностью интерактивной настройки параметров освещения. Программа реализована с графическим интерфейсом и поддерживает как двухмерную визуализацию, так и пространственное отображение результата.

\emph{Цель настоящей работы} -- разработка программно-информационной системы, применимой для создания карт нормалей из 2D-изображений и их интерактивной визуализации.

Для достижения цели необходимо решить следующие \emph{задачи}:
\begin{itemize}
	\item провести анализ предметной области и аналогичных программных решений;
	\item разработать архитектуру программной системы;
	\item реализовать алгоритм генерации карты нормалей;
	\item обеспечить визуализацию с возможностью настройки освещения;
	\item провести тестирование и анализ полученных результатов.
\end{itemize}

\emph{Структура и объем работы.} Отчет состоит из введения, 4 разделов основной части, заключения, списка использованных источников, 2 приложений. Текст выпускной квалификационной работы равен \formbytotal{lastpage}{страниц}{е}{ам}{ам}.

\emph{Во введении} сформулирована цель работы, поставлены задачи разработки, описана структура работы, приведено краткое содержание каждого из разделов.

\emph{В первом разделе} проводится анализ предметной области, рассматриваются существующие подходы к генерации карт нормалей и методы визуализации изображений с применением освещения.

\emph{Во втором разделе} формулируются требования к разрабатываемой программной системе, описываются функциональные возможности и сценарии использования.

\emph{В третьем разделе} описывается реализация программной системы: архитектура, структура классов, интерфейс, а также алгоритмы генерации нормалей и визуализации.

\emph{В четвертом разделе} проводится тестирование системы, анализируются результаты работы при различных параметрах, оценивается эффективность и возможные области применения.

В заключении излагаются основные выводы и направления дальнейшего развития.

В приложении А представлен графический материал, иллюстрирующий работу программы.
  
В приложении Б приведены фрагменты исходного кода.
