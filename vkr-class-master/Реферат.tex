\abstract{РЕФЕРАТ}

Объем работы равен \formbytotal{lastpage}{страниц}{е}{ам}{ам}. Работа содержит \formbytotal{figurecnt}{иллюстраци}{ю}{и}{й}, \formbytotal{tablecnt}{таблиц}{у}{ы}{}, \arabic{bibcount} библиографических источников и \formbytotal{числоПлакатов}{лист}{}{а}{ов} графического материала. Количество приложений -- 2. Графический материал представлен в приложении А. Фрагменты исходного кода представлены в приложении Б.

Ключевые слова: карта нормалей, визуализация, освещение, 2D, 3D, текстура, интерфейс, нормали, визуальный анализ, изображение, скалярное произведение, интерактивная система, глубина, интенсивность, освещённость, графика, градиентный оператор, освещение в реальном времени.

Объектом разработки является программно-информационная система, предназначенная для создания карт нормалей из 2D-изображений и их интерактивной визуализации с целью дальнейшего применения в области компьютерной графики и разработки игр.

Целью выпускной квалификационной работы является создание программной системы, позволяющей визуализировать рельеф изображения на основе карты нормалей в режиме реального времени, обеспечивая гибкость настройки параметров генерации и освещения.

В рамках разработки были выделены основные модули: загрузка изображений, генерация карты нормалей с использованием градиентных операторов, визуализация освещённой текстуры в 2D и 3D-режимах, а также интерфейсный модуль для управления параметрами генерации и визуализации. Система позволяет интерактивно управлять параметрами генерации, а также источником освещения, что обеспечивает гибкость в анализе рельефа текстур и их использовании в игровых или визуализирующих приложениях.

\selectlanguage{english}
\abstract{ABSTRACT}
  
The volume of work is \formbytotal{lastpage}{page}{}{s}{s}. The work contains \formbytotal{figurecnt}{illustration}{}{s}{s}, \formbytotal{tablecnt}{table}{}{s}{s}, \arabic{bibcount} bibliographic sources and \formbytotal{числоПлакатов}{sheet}{}{s}{s} of graphic material. The number of applications is 2. The graphic material is presented in annex A. Fragments of the source code are presented in annex B.

List of keywords: normal map, visualization, lighting, 2D, 3D, texture, interface, normals, visual analysis, image, scalar product, interactive system, depth, intensity, illumination, graphics, gradient operator, real-time lighting.

The object of the development is a software and information system designed to create normal maps from 2D images and their interactive visualization for further application in the field of computer graphics and game development.

The purpose of the final qualifying work is to create a software system that allows you to visualize the relief of an image based on a normal map in real time, providing flexibility in setting generation and lighting parameters.

As part of the development, the main modules were highlighted: image loading, normal map generation using gradient operators, visualization of illuminated textures in 2D and 3D modes, as well as an interface module for controlling generation and visualization parameters. The system allows you to interactively control the generation parameters, as well as the lighting source, which provides flexibility in analyzing the relief of textures and their use in gaming or visualization applications.
\selectlanguage{russian}
