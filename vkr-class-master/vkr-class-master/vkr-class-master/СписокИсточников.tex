\addcontentsline{toc}{section}{СПИСОК ИСПОЛЬЗОВАННЫХ ИСТОЧНИКОВ}

\begin{thebibliography}{9}

    \bibitem{gambetta2021}
    Гамбетта, Г. Компьютерная графика с нуля / Г. Гамбетта. – Сан-Франциско: No Starch Press, 2021. – 248 с. – ISBN 978-1-7185-0076-1. – Текст: непосредственный.
    
    \bibitem{lukacs2020}
    Лукач, Р.; Платаниотис, К. Н. (ред.) Обработка цветных изображений: методы и приложения / под ред. Р. Лукача, К. Н. Платаниотиса. – Бока-Ратон: CRC Press, 2020. – 560 с. – ISBN 978-0-8493-7491-4. – Текст: непосредственный.
    
    \bibitem{reed2020}
    Рид, Т. Р. (ред.) Цифровая последовательность изображений: методы и приложения / под ред. Т. Р. Рида. – Нью-Йорк: Springer, 2020. – 400 с. – ISBN 978-0-387-94854-6. – Текст: непосредственный.
    
    \bibitem{thalmann2021}
    Магненат-Тальманн, Н. (ред.) Достижения в компьютерной графике: материалы 38-й Международной конференции по компьютерной графике CGI 2021 / под ред. Н. Магненат-Тальманн. – Чам: Springer, 2021. – 392 с. – ISBN 978-3-030-89028-5. – Текст: непосредственный.
    
    \bibitem{martins2024}
    Мартинс, К. Обработка изображений и компьютерное зрение / К. Мартинс. – Амазон: Amazon KDP, 2024. – 150 с. – ISBN 978-1-23456-789-0. – Текст: непосредственный.
    
    \bibitem{zhang2017}
    Чжан, Ю. Обработка изображений. Том 1 / Ю. Чжан. – Берлин: De Gruyter, 2017. – 400 с. – ISBN 978-3-11-052411-6. – Текст: непосредственный.
    
    \bibitem{ansari2024}
    Ансари, И. А.; Баджадж, В. Обработка изображений с использованием Python: практический подход / И. А. Ансари, В. Баджадж. – Лондон: IOP Publishing, 2024. – 350 с. – ISBN 978-0-7503-5924-5. – Текст: непосредственный.
    
    \bibitem{bruno2024}
    Бруно, А.; Маццео, П. Л.; Куэвас, Ф. Цифровая обработка изображений: последние достижения и приложения / А. Бруно, П. Л. Маццео, Ф. Куэвас. – Лондон: IOP Publishing, 2024. – 234 с. – ISBN 978-0-85466-491-7. – Текст: непосредственный.
    
    \bibitem{umbaugh2022}
    Умбау, С. Э. Цифровая обработка и анализ изображений: улучшение, восстановление и сжатие цифровых изображений / С. Э. Умбау. – Эдвардсвилл: CRC Press, 2022. – 928 с. – ISBN 978-1-032-07130-5. – Текст: непосредственный.
    
    \bibitem{tyagi2021}
    Тьяги, В. Понимание цифровой обработки изображений / В. Тьяги. – Лондон: Taylor \& Francis, 2021. – 368 с. – ISBN 978-0-367-78082-1. – Текст: непосредственный.
    
    \bibitem{matveev2023}
    Матвеев, А. И. Цифровая обработка изображений в OpenCV. Практикум: учебное пособие. – 2-е изд., стереотип. – СПб.: Лань, 2023. – 104 с. – ISBN 978-5-507-46249-0. – Текст: непосредственный.
    
    \bibitem{baskar2023}
    Баскар, А.; Раджаппа, М.; Васудеван, Ш. К.; Муругеш, Т. С. Цифровая обработка изображений / А. Баскар, М. Раджаппа, Ш. К. Васудеван, Т. С. Муругеш. – Лондон: Chapman \& Hall, 2023. – 208 с. – ISBN 978-1-032-10857-5. – Текст: непосредственный.
    
    \bibitem{yane2021}
    Яне, Б. Цифровая обработка изображений: Концепции, алгоритмы и научные приложения / Б. Яне. – Берлин: Springer, 2021. – 600 с. – ISBN 978-3-662-21818-1. – Текст: непосредственный.
    
    \bibitem{shafik2020}
    Шафик, С.; Машкур, А.; Майр-Дорн, К.; Эгед, А. Машинное обучение в разработке программного обеспечения: систематический обзор / С. Шафик и др. – arXiv, 2020. – 30 с. – arXiv:2005.13299. – Текст: непосредственный.
    
    \bibitem{jehangiri2024}
    Жехангири, З. М.; Шахзад, М.; Хан, У. (ред.) Цифровая обработка изображений: передовые технологии и приложения / под ред. З. М. Жехангири, М. Шахзад, У. Хан. – Базель: MDPI, 2024. – 348 с. – ISBN 978-3-7258-1825-9. – Текст: непосредственный.
    
    \bibitem{farinella2020}
	Фаринелла, Дж. М. и др. (ред.) Компьютерное зрение, визуализация и теория компьютерной графики: материалы 15-й Международной конференции VISIGRAPP 2020 / под ред. Дж. М. Фаринелла и др. – Чам: Springer, 2020. – 392 с. – ISBN 978-3-030-94893-1. – Текст: непосредственный.
    
    \bibitem{matiz2020}
    Мэтиз, Э. Изучаем Python: программирование игр, визуализация данных, веб-приложения / Э. Мэтиз. – СПб.: Питер, 2020. – 512 с. – ISBN 978-5-4461-0923-4. – Текст: непосредственный.
    
    \bibitem{johnson2020}
    Джонсон, Д. Проектирование с учетом пользователя: простое руководство по пониманию принципов проектирования пользовательского интерфейса / Д. Джонсон. – Амстердам: Morgan Kaufmann, 2020. – 250 с. – ISBN 978-0-12-818202-4. – Текст: непосредственный.
    
    \bibitem{ramalho2022}
    Рамальо, Л. Глубокое освоение Python (2-е издание) / Л. Рамальо. – Себастопол: O’Reilly Media, 2022. – 1012 с. – ISBN 978-1-4920-4604-0. – Текст: непосредственный.
    
    \bibitem{dawson2021}
    Доусон, М. Программируем на Python / М. Доусон. – М.: Вильямс, 2021. – 400 с. – ISBN 978-5-8459-7890-1. – Текст: непосредственный.
    
    \bibitem{tidwell2020}
    Тидвелл, Дж.; Брюэр, К.; Валенсия, А. Проектирование интерфейсов (3-е издание) / Дж. Тидвелл, К. Брюэр, А. Валенсия. – Себастопол: O’Reilly Media, 2020. – 560 с. – ISBN 978-1-4920-6832-5. – Текст: непосредственный.
    
    \bibitem{prokhorenok2021}
    Прохоренок, Н. А.; Дронов, В. А. Python 3 и PyQt 5. Разработка приложений. – 2-е изд., перераб. и доп. – СПб.: БХВ-Петербург, 2021. – 832 с. – ISBN 978-5-9775-3978-4. – Текст: непосредственный.
    
    \bibitem{cooper2020}
    Купер, А.; Райман, Р.; Кронин, Д. Внимание к интерфейсу: основы проектирования взаимодействия (4-е издание) / А. Купер, Р. Райман, Д. Кронин. – Индианаполис: Wiley, 2020. – 720 с. – ISBN 978-1-118-76457-3. – Текст: непосредственный.
    
    \bibitem{ravichandiran2020}
    Равичандиран, С. Глубокое обучение с подкреплением на Python / С. Равичандиран. – Москва: ДМК Пресс, 2020. – 450 с. – ISBN 978-5-97060-789-2. – Текст: непосредственный.
    
    \bibitem{talipov2020}
    Талипов, С. Н. Программирование на Python3 с PyQt5. – Екатеринбург: Уральское издательство, 2020. – 155 с. – ISBN 978-5-04-350119-6. – Текст: непосредственный.
    
    \bibitem{kaehler2019}
    Кэлер, А.; Брэдски, Г. Изучаем OpenCV 3. – М.: ДМК Пресс, 2019. – 640 с. – ISBN 978-5-97060-471-7. – Текст: непосредственный.
    
    \bibitem{russ2020}
    Русс, Дж. К.; Нил, Ф. Б. Справочник по обработке изображений / Дж. К. Русс, Ф. Б. Нил. – Лондон: CRC Press, 2020. – 1032 с. – ISBN 978-1-4822-3959-0. – Текст: непосредственный.
    
    \bibitem{day2021}
    Дей, С. Мастер-класс по обработке изображений с использованием Python: 50+ решений и техник / С. Дей. – Нью-Дели: BPB Publications, 2021. – 450 с. – ISBN 978-93-89898-64-1. – Текст: непосредственный.
    
    \bibitem{bhuiyan2020}
    Бхуйян, М. К. Компьютерное зрение и обработка изображений: основы и приложения / М. К. Бхуйян. – Бока-Ратон: CRC Press, 2020. – 468 с. – ISBN 978-0-8153-7084-0. – Текст: непосредственный.
    
    \bibitem{koltsov2024}
    Кольцов, Д. М. Python. Полное руководство. – 2-е изд., испр. и обнов. – М.: Наука и техника, 2024. – 512 с. – ISBN 978-5-94387-270-9. – Текст: непосредственный.

    \bibitem{Bueno2018} Буэно Гарсия, Г.; Суарес, О. Д.; Эспиноса Аранда, Х. Л. Обработка изображений с помощью OpenCV. – М.: ДМК Пресс, 2018. – 352 с. – ISBN 978-5-97060-387-1. – Текст: непосредственный.
    
    \bibitem{Shitov2023} Шитов, В. Н.; Успенский, К. Е. Проектирование и разработка интерфейсов пользователя. Учебное пособие. ФГОС СПО / В. Н. Шитов, К. Е. Успенский. – Москва: Кнорус, 2023. – 296 с. – ISBN 978-5-406-10392-0. – Текст: непосредственный.
    
    \bibitem{Cooper2022} Купер, А.; Рейман, Р.; Кронин, Д. Интерфейс. Основы проектирования взаимодействия / А. Купер, Р. Рейман, Д. Кронин. – Санкт-Петербург: Питер, 2022. – 720 с. – ISBN 978-5-496-01718-3. – Текст: непосредственный.
    
    \bibitem{Plachta2024} Плахта, М. Грокаем функциональное программирование / М. Плахта. – Москва: ДМК Пресс, 2024. – 400 с. – ISBN 978-5-97060-456-3. – Текст: непосредственный.
    
    \bibitem{Rafgarden2021} Рафгарден, Т. Совершенный алгоритм. Графовые алгоритмы и структуры данных / Т. Рафгарден. – Москва: ДМК Пресс, 2021. – 400 с. – ISBN 978-5-97060-322-1. – Текст: непосредственный.
    
    \bibitem{Gorelick2020} Горелик, М.; Озсвальд, И. Высокопроизводительный Python (2-е издание) / М. Горелик, И. Озсвальд. – Себастопол: O’Reilly Media, 2020. – 466 с. – ISBN 978-1-4920-5891-3. – Текст: непосредственный.
    
    \bibitem{Nguyen2022} Нгуен, Куок Хоанг. Продвинутое программирование на Python (2-е издание) / К. Нгуен. – Бирмингем: Packt Publishing, 2022. – 400 с. – ISBN 978-1-80181-401-0. – Текст: непосредственный.
    
    \bibitem{Hillard2020} Хиллард, Д. Практика профессионального программиста на Python / Д. Хиллард. – Шелтер-Айленд: Manning Publications, 2020. – 248 с. – ISBN 978-1-61729-608-6. – Текст: непосредственный.
    
    \bibitem{Johnson2020} Джонсон, Д. Проектирование с учетом пользователя: простое руководство по пониманию принципов проектирования пользовательского интерфейса / Д. Джонсон. – Амстердам: Morgan Kaufmann, 2020. – 250 с. – ISBN 978-0-12-818202-4. – Текст: непосредственный.
    
    \bibitem{Jablonski2020} Яблонски, Дж. Законы UX: использование психологии для проектирования лучших продуктов и сервисов / Дж. Яблонски. – [Место издания не указано]: O’Reilly Media, 2020. – 168 с. – ISBN 978-1-4920-6842-4. – Текст: непосредственный.
    
    \bibitem{Woelter2020} Вёлтер, М. Анализ предметной области: сбор, осмысление и валидация / М. Вёлтер. – [Место издания не указано]: [Издательство не указано], 2020. – 350 с. – ISBN [не указан]. – Текст: непосредственный.
    
    \bibitem{Unhelkar2020} Унхелкар, Б. Разработка программного обеспечения с использованием UML / Б. Унхелкар. – Нью-Йорк: CRC Press, 2020. – 350 с. – ISBN 978-0-367-65738-3. – Текст: непосредственный.
    
    \bibitem{Yamamoto2020} Ямамото, С.; Мори, Х. (ред.) Человеко-компьютерное взаимодействие и управление информацией: проектирование информации / под ред. С. Ямамото, Х. Мори. – Шампейн: Springer, 2020. – 600 с. – ISBN 978-3-030-50020-7. – Текст: непосредственный.
    
    \bibitem{Vlashchin2021} Влашчин, С. Функциональное моделирование предметной области / С. Влашчин. – [Место издания не указано]: [Издательство не указано], 2021. – 400 с. – ISBN [не указан]. – Текст: непосредственный.
    
    \bibitem{Chen2024} Чэнь, С.; Би, Цз.; Лю, Цз.; Пэн, Б. и др. Глубокое и машинное обучение: структуры данных на Python и математические основы / С. Чэнь и др. – arXiv, 2024. – 50 с. – arXiv:2401.00001. – Текст: непосредственный.
    
    \bibitem{Ambrosio2022} Амбросио, П. (ред.) Применения цифровой обработки изображений / под ред. П. Амбросио. – Ильеус: IntechOpen, 2022. – 126 с. – ISBN 978-1-83969-794-4. – Текст: непосредственный.
    
    \bibitem{Downey2021} Дауни, А. Б. Основы Python. Научитесь мыслить, как программист / А. Б. Дауни. – М.: Манн, Иванов и Фербер, 2021. – 304 с. – ISBN 978-5-00146-798-4. – Текст: непосредственный.
    
    \bibitem{Karmagi2020} Карамаги, Р. М. Продвинутое программирование на Python / Р. М. Карамаги. – [Место издания не указано]: Independently Published, 2020. – 800 с. – ISBN 979-8699711352. – Текст: непосредственный.
    
    \bibitem{Sweigart2021} Свейгарт, Э. Автоматизация рутинных задач с помощью Python / Э. Свейгарт. – СПб.: Питер, 2021. – 400 с. – ISBN 978-5-4461-1234-5. – Текст: непосредственный.
    
    \bibitem{Oaks2022} Оукс, Б. Дж.; Фамелис, М.; Сахрауи, Х. Построение предметно-ориентированных рабочих процессов машинного обучения / Б. Дж. Оукс и др. – arXiv, 2022. – 25 с. – arXiv:2203.08638. – Текст: непосредственный.
    
    
\end{thebibliography}
