\section*{ЗАКЛЮЧЕНИЕ}
\addcontentsline{toc}{section}{ЗАКЛЮЧЕНИЕ}

Современные технологии обработки изображений и генерации нормалей находят всё более широкое применение в таких сферах, как компьютерная графика, визуализация, моделирование и разработка игр. В ходе выполнения данной выпускной квалификационной работы была спроектирована и реализована программная система, обеспечивающая построение карт нормалей из 2D-изображений и их интерактивную визуализацию.

Основные результаты работы:

\begin{enumerate}
	\item Выполнен анализ предметной области, охватывающей задачи генерации карт нормалей и их применения в визуальных приложениях.
	\item Разработана концептуальная модель программного решения, определены функциональные и нефункциональные требования к системе.
	\item Спроектирован и реализован пользовательский интерфейс, обеспечивающий интуитивную работу с изображениями, настройку параметров генерации и визуализации.
	\item Внедрены режимы двухмерной и трёхмерной визуализации с возможностью симуляции освещения на основе нормалей.
	\item Проведено тестирование программной системы, подтверждающее её работоспособность и соответствие заданным требованиям.
\end{enumerate}

Поставленные задачи были успешно решены. Результатом работы стала программа с графическим интерфейсом, позволяющая пользователю не только генерировать карты нормалей из изображений, но и анализировать их в интерактивном режиме с управляемыми параметрами освещения.

Полученные результаты демонстрируют практическую ценность выполненной разработки и подтверждают успешное достижение поставленных в начале проекта целей.

