\abstract{РЕФЕРАТ}

Объем работы равен \formbytotal{lastpage}{страниц}{е}{ам}{ам}. Работа содержит \formbytotal{figurecnt}{иллюстраци}{ю}{и}{й}, \formbytotal{tablecnt}{таблиц}{у}{ы}{}, \arabic{bibcount} библиографических источников и \formbytotal{числоПлакатов}{лист}{}{а}{ов} графического материала. Количество приложений -- 2. Графический материал представлен в приложении А. Фрагменты исходного кода представлены в приложении Б.

Ключевые слова: карта нормалей, визуализация, освещение, 2D, 3D, текстура, интерфейс, нормали, визуальный анализ, изображение, скалярное произведение, интерактивная система, глубина, интенсивность, освещённость, графика, градиентный оператор, освещение в реальном времени.

Объектом разработки является программно-информационная система, обеспечивающая создание карт нормалей из загруженных 2D-изображений и интерактивную визуализацию с гибкими настройками параметров, ориентированная на применение в анализе и обработке изображений для игровых приложений.

Целью выпускной квалификационной работы является создание программно-информационной системы, ориентированной на создание специализированных 2D-изображений -- карт нормалей -- из исходных растровых изображений, с возможностью интерактивного просмотра и настройки параметров генерации.

В рамках разработки были выделены основные модули: загрузка изображений, генерация карты нормалей с использованием градиентных операторов, визуализация освещённой текстуры в 2D и 3D-режимах, а также интерфейсный модуль для управления параметрами генерации и визуализации. Система позволяет интерактивно управлять параметрами генерации, а также источником освещения, что обеспечивает гибкость в анализе рельефа текстур и их использовании в игровых или визуализирующих приложениях.

\selectlanguage{english}
\abstract{ABSTRACT}
  
The volume of work is \formbytotal{lastpage}{page}{}{s}{s}. The work contains \formbytotal{figurecnt}{illustration}{}{s}{s}, \formbytotal{tablecnt}{table}{}{s}{s}, \arabic{bibcount} bibliographic sources and \formbytotal{числоПлакатов}{sheet}{}{s}{s} of graphic material. The number of applications is 2. The graphic material is presented in annex A. Fragments of the source code are presented in annex B.

List of keywords: normal map, visualization, lighting, 2D, 3D, texture, interface, normals, visual analysis, image, scalar product, interactive system, depth, intensity, illumination, graphics, gradient operator, real-time lighting.

The object of the development is a software and information system that provides the creation of normal maps from uploaded 2D images and interactive visualization with flexible parameter settings, focused on application in image analysis and processing for gaming applications.

The purpose of the final qualifying work is to create a software and information system focused on creating specialized 2D images -- normal maps -- from the original raster images, with the ability to interactively view and configure generation parameters.

As part of the development, the main modules were highlighted: image loading, normal map generation using gradient operators, visualization of illuminated textures in 2D and 3D modes, as well as an interface module for controlling generation and visualization parameters. The system allows you to interactively control the generation parameters, as well as the lighting source, which provides flexibility in analyzing the relief of textures and their use in gaming or visualization applications.
\selectlanguage{russian}
